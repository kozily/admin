\documentclass[a4paper,11pt]{article}

%%%%%%%%%%%%%%%%%%%%%%%%%%%%%%%%%%%%%%%%%%%%%%%%%%%%%%%%%%%%%%%%%%%%%%%%
% Paquetes utilizados
%%%%%%%%%%%%%%%%%%%%%%%%%%%%%%%%%%%%%%%%%%%%%%%%%%%%%%%%%%%%%%%%%%%%%%%%
% Soporte para el lenguaje español
\usepackage{textcomp}
\usepackage[utf8]{inputenc}
\usepackage[T1]{fontenc}
\usepackage[spanish]{babel}

% Configuración de página
\usepackage[lmargin=1in, rmargin=1in]{geometry}

% Graficos
\usepackage{graphicx}

% Encabezados y pies de pagina
\usepackage{fancyhdr}
\setlength{\headheight}{15.2pt}
\pagestyle{fancy}
\fancyhf{}
\lhead{Kozily}
\rhead{Propuesta de Proyecto}
\cfoot{\thepage}

% Tablas
\usepackage{longtable}

\begin{document}

%%%%%%%%%%%%%%%%%%%%%%%%%%%%%%%%%%%%%%%%%%%%%%%%%%%%%%%%%%%%%%%%%%%%%%%%
% Titulo
%%%%%%%%%%%%%%%%%%%%%%%%%%%%%%%%%%%%%%%%%%%%%%%%%%%%%%%%%%%%%%%%%%%%%%%%

\thispagestyle{empty}

\begin{titlepage}

  \newcommand{\HRule}{\rule{\linewidth}{0.5mm}}
  \newenvironment{bottompar}{\par\vspace*{\fill}}{\clearpage}

  \center

  \textsc{\LARGE Universidad de Buenos Aires}\\[0.5cm]
  \textsc{\Large Facultad de Ingeniería}\\[1.5cm]

  \includegraphics[scale=0.5]{assets/logo.png}\\[1cm]


  \textsc{\large (75.99) Trabajo Profesional}\\[0.25cm]
  {\huge \bfseries Propuesta de Proyecto} \\[0.4cm]
  \HRule \\[0.4cm]
  {\huge \bfseries Kozily} \\[0.4cm]
  Un entorno de desarrollo de Oz para estudiantes
  \HRule

  \begin{bottompar}
    \begin{minipage}[t]{.45\linewidth}
      \begin{flushleft}
        {\bfseries Autores:}

        Arana, Andrés          - P. 86.203

        Piano, Sergio          - P. 85.191
      \end{flushleft}
    \end{minipage}
    \hfill
    \begin{minipage}[t]{.45\linewidth}
      \begin{flushright}
        {\bfseries Tutor:}

        Rosita Wachenchauzer
      \end{flushright}
    \end{minipage}
  \end{bottompar}

\end{titlepage}

% ----------------------------------------------------------------------
% Documento
% ----------------------------------------------------------------------
\begin{abstract}

  El presente documento constituye una propuesta de proyecto enmarcado en el
  contexto de un trabajo profesional para la carrera de Ingeniería en
  Informática en la Facultad de Ingeniería de la Universidad de Buenos Aires.
  Se propone desarrollar un entorno de desarrollo para el lenguaje de
  programación Oz orientado a apoyar el aprendizaje de los diferentes
  paradigmas que soporta el mismo, y que constituye su principal caso de uso.

\end{abstract}
\clearpage

\tableofcontents
\clearpage

\section{Objetivo}

\subsection{Introducción}

Oz es un lenguaje multiparadigma diseñado por Gert Smolka en la Université
Catholique de Louvain en 1991, específicamente pensado para enseñar conceptos
de programación a alumnos universitarios. Fue desarrollado posteriormente en
1996 por Peter Van Roy y Seif Haridi de Swedish Institute of Computer Science
en conjunto con un manual de estudio canónico: Concepts, Techniques and Models
of Computer Programming. Finalmente, en 2005 se constituyó una entidad propia,
The Mozart Board, para coordinar los esfuerzos de desarrollo del lenguaje, su
especificación y entorno de ejecución.

Si bien la visión fundacional de The Mozart Board es ampliar el espectro de
aplicabilidad del lenguaje, en la actualidad la mayoría de los casos de uso del
mismo se nucléan alrededor de la idea original que sustenta su diseño: como
herramienta para enseñar a alumnos universitarios diversos conceptos, técnicas
y modelos de programación. 

En Argentina, tanto la Facultad de Ingeniería de la Universidad de Buenos Aires
cómo algunas universidades tecnológicas de Córdoba enmarcan la enseñanza de
paradigmas y lenguajes de programación en el enfoque de Concepts, Techniques
and Models of Computer Programming, y por consiguiente, usando Oz como lenguaje
de estudio.

\subsection{Concepts, Techniques and Models of Computer Programming}

El manual de referencia de Peter Van Roy enmarca una metodología incremental de
enseñanza de los diferentes paradigmas, conceptos y técnicas de programación.
En el mismo, el autor desarrolla un marco teórico de ejecución, la
\emph{máquina abstracta de ejecución}, un sistema matemático abstracto que
define la ejecución de un programa en base a un conjunto minimalista de
conceptos: un stack de ejecución con diferentes instrucciones, un entorno de
ejecución que asigna identificadores a celdas de memoria y una memoria que
almacena valores en celdas. Las diferentes técnicas, conceptos y modelos que
son examinados a lo largo del libro se definen en base a los efectos de las
instrucciones constituyentes sobre la máquina abstracta, de acuerdo a la
semántica operacional que se define iterativamente en cada capítulo.

Este enfoque es sumamente efectivo dado que le permite al alumno comprender el
mecanismo de abstracción que fundamenta el aprendizaje de la programación: el
lenguaje de programación es una interfase con una máquina subyacente, y su
diseño está a su vez estratificado en niveles de abstracción sobre una
semántica base (procedimientos, funciones, functores, objetos y otros). Aún
más, este conocimiento es entendido sin necesidad de comprender la operativa
física de la máquina subyacente real, lo que requiere contar con experiencia en
ciencias básicas (física, química) y aplicadas (electrónica).

Por último, Oz expone una variedad de paradigmas que permite al docente diseñar
un currículo que incorpore conceptos básicos en cada uno de ellos. Esto es
especialmente importante hoy en día, donde las arquitecturas basadas en
microservicios buscan explotar la separación de los componentes de un sistema
informático en microsistemas desarrollados con diferentes lenguajes de
programación de manera de aplicar paradigmas y técnicas más efectivas en cada
dominio.

\subsection{Herramientas existentes}

Actualmente existe un único entorno de desarrollo, administrado y mantenido por
The Mozart Board. El entorno es extremadamente completo, pero tiene una gran
debilidad: debido a que la misión fundacional de The Mozart Board es expandir
el campo de aplicabilidad del lenguaje Oz más allá del nicho de la enseñanza,
suele ser complejo para ser abordado por alguien que recién se está iniciando
en el campo de la programación.

En particular, podemos destacar las siguientes dificultades asociadas al uso de
este entorno en el marco educativo:

\begin{enumerate}

  \item El sistema está basado en GNU Emacs. Si bien este constituye un excelente
    entorno de desarrollo en general, tiene una curva de aprendizaje desfavorable y
    requiere una inversión de tiempo considerable para poder operarlo.

  \item Muchos de los conceptos subyacentes a la máquina de ejecución abstracta
    están escondidos dentro del entorno de ejecución, y el entorno de desarrollo
    no provee herramientas para acceder a esta información.

  \item El sistema es una aplicación binaria de escritorio, lo que significa que
    requiere un proceso de instalación que constituye nuevamente una barrera
    adicional para el neófito, sobre todo si se tiene en cuenta que no existen
    paquetes binarios actualizados para las diferentes plataformas que puede
    tener el alumno.

  \item El sistema no puede ser ejecutado sobre un dispositivo móvil, por lo que
    requiere que el alumno cuente con una computadora para poder ejecutar,
    evaluar e interactuar con el software, cuando a priori podría utilizar
    cualquiera de los dispositivos móviles que suelen estar más al alcance de la
    mano.

\end{enumerate}

\subsection{Propuesta}

En base a lo descripto anteriormente, se propone implementar un nuevo entorno
de desarrollo cuya misión sea facilitar la comprensión de las características
operativas del entorno de ejecución de Oz.

\subsection{Requerimientos funcionales}

\subsection{Requerimientos no funcionales}

\section{Alcance}

\section{Herramientas de desarrollo}

\section{Metodología}

\section{Cronograma}

\section{Curriculum Vitae}

\subsection{Andrés Gastón Arana}

\subsubsection{Datos personales}

\noindent \begin{tabular}{l l l l}
  \textbf{Nombre} & Andrés Gastón Arana & \textbf{Teléfono} & +54 9 11 3162-7877\\
  \textbf{Padrón} & 86.203              & \textbf{Email}    & and2arana@gmail.com \\
\end{tabular}

\subsubsection{Educación}

\noindent \textbf{Ingeniería en Informática}

\noindent\emph{2004 - Actual}

\noindent Cursando Ingeniería en Informática en la Facultad de Ingeniería de la
Universidad de Buenos Aires. \\

\noindent \textbf{Polimodal con orientación en adm. y gestión de empresas}

\noindent\emph{2000 - 2003}

\noindent Bachiller con orientación en adm. y gestión de empresas en el colegio
Mariano Moreno.

\subsubsection{Experiencia laboral}

\noindent \textbf{Socio fundador en Recompensa.mobi}

\noindent \emph{2014 - Actual}

\noindent Participo actualmente en el desarrollo, comercialización e
implementación de una plataforma de registro de eventos asociados con los
procesos de ventas y customer service en PyMes. Desarrollo diversos roles,
principalmente relacionados al relevamiento de requerimientos, diseño,
conceptualización de arquitectura e implementación. \\

\noindent \textbf{Desarrollador en Global Fishing Watch}

\noindent \emph{2015 - Actual}

\noindent Desempeño tareas de consultoría como desarrollador en Global Fishing
Watch, una ONG que realiza monitoreo satelital de barcos de pesca. Involucrado
en el procesamiento de eventos AIS para detectar patrones de pesca, que incluye
técnicas probabilísticas, heurísticas y de machine learning aplicados a
100.000.000 de registros por día, así como también infraestructura de
visualización de la información generada. \\

\noindent \textbf{Líder técnico en Asante}

\noindent \emph{2011 - 2014}

\noindent Tareas de líder técnico para diversos proyectos en Asante. Trabajé
principalmente coordinando un equipo de 5 desarrolladores y realizando tareas
de desarrollo para varios proyectos que involucraban la generación de
aplicaciones para dispositivos móviles Android, iOS y web nativo, incluyendo un
backend administrativo en Ruby. \\

\noindent \textbf{Líder técnico en PlayPhone}

\noindent \emph{2011 - 2011}

\noindent Desarrollé tareas de líder técnico en PlayPhone, trabajando
principalmente en la coordinación e implementación de sistemas de backend de
una plataforma de servicios para aplicaciones móviles como autorización,
gestión de pagos y microtransacciones en .NET. \\

\noindent \textbf{Desarrollador en LeaseTrader}

\noindent \emph{2010 - 2011}

\noindent Tareas de desarrollo bajo la plataforma .NET para un sitio de
compra-venta de leases de autos en estados unidos, incluyendo principalmente el
desarrollo y testing de nuevos features, así como el diseño, arquitectura e
implementación de nuevas plataformas de soporte de la compañía. \\

\noindent \textbf{Desarrollador en Sabarasa Entertainment}

\noindent \emph{2009 - 2010}

\noindent Desarrollo de videojuegos para consolas Wii en lenguajes de bajo
nivel, principalmente C++, incluyendo tareas de modelado matemático y
optimización de bajo nivel para la plataforma, incluyendo sistemas de
procesamiento en tiempo real de entrada y subsistemas gráficos. \\

\noindent \textbf{Desarrollador en Globant}

\noindent \emph{2008 - 2009}

\noindent Tareas de desarrollo en múltiples clientes, pero principalmente en un
proyecto de modernización para una plataforma de servicios y distribución de
contenido en sistemas móviles que incluía la reingeniería de un sistema de
facturación y autorización en un stack JVM, Spring y JBoss. \\

\noindent \textbf{Desarrollador en EDS}

\noindent \emph{2007 - 2008}

\noindent Tareas de desarrollo en un equipo chico a cargo de la modernización
de la infraestructura de procesamiento de aprobación de créditos para la compra
de autos en GMAC. El proyecto consistía en la migración de una infraestructura
de más de 10 años desarrollada en C++ a una plataforma .NET. \\

\noindent \textbf{Desarrollador en LanTec IT}

\noindent \emph{2007 - 2008}

\noindent Tareas de desarrollo en el proyecto de QuickPass, un emprendimiento de procesamiento biométrico para control de acceso y registro de horas laborales. Componentes desarrollados en C++ para interactuar con el hardware, y en .NET para el resto de las aplicaciones. Desempeñé también tareas de electrónica para construir dispositivos de toma de huellas dactilares en base a componentes importados. \\

\noindent \textbf{Desarrollador en Sistemas Bejerman}

\noindent \emph{2004 - 2007}

\noindent Tareas de desarrollo los módulos de contabilidad, balances, recursos
humanos y sueldos y jornales del ERP que constituía el producto principal de la
empresa, una plataforma .NET. Adicionalmente codificaba los instaladores del
sistema, bajo la tecnología de instalación MSI, con binarios precompilados en
Delphi. \\

\subsubsection{Habilidades}

\begin{enumerate}

  \item Experiencia en trabajo en equipos de desarrollo bajo diversas
    metodologías, tanto ágiles (Kanban, Scrum) como tradicionales.

  \item Experiencia en trabajo bajo condiciones de continuous integration y
    continuous delivery.

  \item Conocimiento avanzado en diversos lenguajes de programación. C++,
    Java, C\#, F\#, Ruby, Python, Clojure, LISP, Haskell y Javascript.

  \item Extensiva experiencia laboral en sistemas distribuídos web, incluyendo
    arquitecturas REST, microservicios sobre HTTP y aplicaciones web.

  \item Experiencia laboral en desarrollo de clientes progresivos sobre
    exploradores móviles y de escritorio, con aplicación de tecnologías como
    HTML5, CSS3, responsive design, progressive enhancement, service workers y
    offline first clients.

  \item Experiencia laboral sobre plataformas y librerías de desarrollo para
    cada entorno, como ser Spring, JUnit, JMock, JBoss para plataformas Java,
    Rails, RSpec, Cucumber para plataformas Ruby, Django para python, Compojure y Ring para
    Clojure, etc..

  \item Experiencia laboral sobre data modeling y data stores, tanto SQL
    (MySQL, Oracle, MSSQL, Postgresql) como noSQL (MongoDB, Google Cloud
    Datastore, Redis).

  \item Experiencia laboral sobre BigData y las tecnologías relacionadas, como
    Spark, Hadoop y Kafka. Incluye aplicación de técnicas de análisis
    heurísticos, probabilísticos y de machine learning (modelos bayesianos y
    redes neuronales).

  \item Experiencia laboral en diversas plataformas de cloud computing, como
    AWS y Google Cloud Platform. Entornos virtualizados y en base a
      contenedores (Docker) con coordinadores como Kubernetes.

\end{enumerate}

\subsubsection{Idiomas}

\begin{enumerate}

  \item Español nativo.

  \item Inglés avanzado. Nivel B en CAE, A en FCE.

  \item Francés principiante. Sin certificación avalada.

\end{enumerate}

\subsubsection{Materias aprobadas}

\begin{longtable}{|l|l|r|l|}
  \hline
  Materia                                       & Fecha      & Nota & Acta              \\
  \hline
  \endhead

  \hline
  \multicolumn{4}{r}{\textit{Continua en la próxima página}} \\
  \endfoot

  \hline
  \endlastfoot

  (6201) Fisica I A                          & 26/07/2005 & 6  & 2-104-20          \\
  (7540) Algoritmos Y Programacion I         & 20/12/2005 & 7  & 17-93-191         \\
  (6108) Algebra II A                        & 21/12/2005 & 5  & 1-150-126         \\
  (6103) Analisis Matematico II A            & 25/07/2006 & 4  & 1-151-98          \\
  (7541) Algoritmos Y Programacion II        & 14/08/2006 & 8  & 17-95-50          \\
  (6301) Quimica                             & 17/08/2006 & 6  & 3-73-44           \\
  (7507) Algoritmos Y Programacion III       & 14/12/2006 & 8  & 17-95-126         \\
  (6203) Fisica II A                         & 21/02/2008 & 7  & 2-106-76          \\
  (6670) Estructura Del Computador           & 12/08/2008 & 6  & 6-136-98          \\
  (7512) Analisis Numerico I                 & 28/12/2009 & 5  & 17-102-191        \\
  (7506) Organizacion De Datos               & 21/02/2011 & 8  & 17-106-24         \\
  (7542) Taller De Programacion I            & 01/03/2011 & 9  & 17-106-69         \\
  (6109) Probabilidad Y Estadistica B        & 21/07/2011 & 5  & 1-159-64          \\
  (6602) Laboratorio                         & 11/08/2011 & 8  & 6-141-34          \\
  (7508) Sistemas Operativos                 & 22/12/2011 & 9  & 17-108-99         \\
  (7112) Estructura De Las Organizaciones    & 08/02/2012 & 7  & 11-153-160        \\
  (7509) Analisis De La Informacion          & 07/08/2012 & 6  & 17-110-1          \\
  (6110) Análisis Matemático III A           & 09/08/2012 & 6  & 1-157-164         \\
  (7114) Modelos Y Optimizacion I            & 12/12/2012 & 8  & 11-155-28         \\
  (6620) Organizacion De Computadoras        & 04/02/2013 & 7  & 6-143-56          \\
  (7514) Lenguajes Formales                  & 25/02/2013 & 9  & 17-111-129        \\
  (7113) Informacion En Las Organizaciones   & 16/07/2013 & 4  & 71-0001053        \\
  (7526) Simulacion                          & 31/07/2013 & 7  & 95-0001278        \\
  (7510) Tecnicas De Diseño                  & 05/08/2013 & 7  & 95-0001318        \\
  (7545) Taller De Desarrollo De Proyectos I & 12/12/2013 & 8  & 95-0001636        \\
  (7515) Base De Datos                       & 26/02/2014 & 5  & 95-0001833        \\
  (6671) Sistemas Gráficos                   & 18/07/2014 & 6  & 86-0001914        \\
  (7552) Taller De Programacion II           & 08/08/2014 & 5  & 95-0002323        \\
  (7559) Tecnicas De Prog. Concurrente I     & 12/12/2014 & 7  & 95-0002481        \\
  (7544) Adm. Y Control De Proy. Inf. I      & 16/12/2014 & 4  & 95-0002510        \\
  (7547) Taller De Desarrollo De Proy. II    & 29/06/2015 & 8  & 95-0002895        \\
  (7140) Leg. Y Ej. Prof. De La Ing. En Inf. & 30/07/2015 & 5  & 71-0002132        \\
  (7531) Teoria De Lenguaje                  & 14/12/2015 & 10 & 95-0003624        \\
  (7550) Introduccion A Los Sist. Int.       & 29/02/2016 & 7  & 95-0003649        \\
  \hline
\end{longtable}

\textbf{Promedio académico}: 6,68

\subsection{Sergio Matías Piano}

\section{Plan de cursada}

\end{document}

