\documentclass[a4paper,11pt]{article}

%%%%%%%%%%%%%%%%%%%%%%%%%%%%%%%%%%%%%%%%%%%%%%%%%%%%%%%%%%%%%%%%%%%%%%%%
% Paquetes utilizados
%%%%%%%%%%%%%%%%%%%%%%%%%%%%%%%%%%%%%%%%%%%%%%%%%%%%%%%%%%%%%%%%%%%%%%%%
% Soporte para el lenguaje español
\usepackage{textcomp}
\usepackage[utf8]{inputenc}
\usepackage[T1]{fontenc}
\usepackage[spanish]{babel}

% Graficos
\usepackage{graphicx}

% Encabezados y pies de pagina
\usepackage{fancyhdr}
\setlength{\headheight}{15.2pt}
\pagestyle{fancy}
\fancyhf{}
\lhead{(75.99) Trabajo Profesional}
\rhead{Plan de Proyecto}
\cfoot{\thepage}

\begin{document}

%%%%%%%%%%%%%%%%%%%%%%%%%%%%%%%%%%%%%%%%%%%%%%%%%%%%%%%%%%%%%%%%%%%%%%%%
% Titulo
%%%%%%%%%%%%%%%%%%%%%%%%%%%%%%%%%%%%%%%%%%%%%%%%%%%%%%%%%%%%%%%%%%%%%%%%

\thispagestyle{empty}

\begin{titlepage}

  \newcommand{\HRule}{\rule{\linewidth}{0.5mm}}
  \newenvironment{bottompar}{\par\vspace*{\fill}}{\clearpage}

  \center

  \textsc{\LARGE Universidad de Buenos Aires}\\[0.5cm]
  \textsc{\Large Facultad de Ingeniería}\\[1.5cm]

  \includegraphics[scale=0.5]{assets/logo.png}\\[1cm]


  \textsc{\large (75.99) Trabajo Profesional}\\[0.25cm]
  {\huge \bfseries Propuesta de Proyecto} \\[0.4cm]
  \HRule \\[0.4cm]
  {\huge \bfseries Kozily} \\[0.4cm]
  Un entorno de desarrollo de Oz para estudiantes
  \HRule

  \begin{bottompar}
    \begin{minipage}[t]{.45\linewidth}
      \begin{flushleft}
        {\bfseries Autores:}

        Arana, Andrés          - P. 86.203

        Piano, Sergio          - P. 85.191
      \end{flushleft}
    \end{minipage}
    \hfill
    \begin{minipage}[t]{.45\linewidth}
      \begin{flushright}
        {\bfseries Tutor:}

        Rosita Wachenchauzer
      \end{flushright}
    \end{minipage}
  \end{bottompar}

\end{titlepage}

% ----------------------------------------------------------------------
% Documento
% ----------------------------------------------------------------------
\begin{abstract}

  El presente documento constituye una propuesta de proyecto enmarcado en el
  contexto de un trabajo profesional para la carrera de Ingeniería en
  Informática en la Facultad de Ingeniería de la Universidad de Buenos Aires.
  Se propone desarrollar un entorno de desarrollo para el lenguaje de
  programación Oz orientado a apoyar el aprendizaje de los diferentes
  paradigmas que soporta el mismo, y que constituye su principal caso de uso.

\end{abstract}
\clearpage

\tableofcontents
\clearpage

\section{Objetivo}

\subsection{Introducción}

Oz es un lenguaje multiparadigma diseñado por Gert Smolka en la Université
Catholique de Louvain en 1991, específicamente pensado para enseñar conceptos
de programación a alumnos universitarios. Fue desarrollado posteriormente en
1996 por Peter Van Roy y Seif Haridi de Swedish Institute of Computer Science
en conjunto con un manual de estudio canónico: Concepts, Techniques and Models
of Computer Programming. Finalmente, en 2005 se constituyó una entidad propia,
The Mozart Board, para coordinar los esfuerzos de desarrollo del lenguaje, su
especificación y entorno de ejecución.

Si bien la visión fundacional de The Mozart Board es ampliar el espectro de
aplicabilidad del lenguaje, en la actualidad la mayoría de los casos de uso del
mismo se nucléan alrededor de la idea original que sustenta su diseño: como
herramienta para enseñar a alumnos universitarios diversos conceptos, técnicas
y modelos de programación. 

En Argentina, tanto la Facultad de Ingeniería de la Universidad de Buenos Aires
cómo algunas universidades tecnológicas de Córdoba enmarcan la enseñanza de
paradigmas y lenguajes de programación en el enfoque de Concepts, Techniques
and Models of Computer Programming, y por consiguiente, usando Oz como lenguaje
de estudio.

\subsection{Concepts, Techniques and Models of Computer Programming}

El manual de referencia de Peter Van Roy enmarca una metodología incremental de
enseñanza de los diferentes paradigmas, conceptos y técnicas de programación.
En el mismo, el autor desarrolla un marco teórico de ejecución, la
\emph{máquina abstracta de ejecución}, un sistema matemático abstracto que
define la ejecución de un programa en base a un conjunto minimalista de
conceptos: un stack de ejecución con diferentes instrucciones, un entorno de
ejecución que asigna identificadores a celdas de memoria y una memoria que
almacena valores en celdas. Las diferentes técnicas, conceptos y modelos que
son examinados a lo largo del libro se definen en base a los efectos de las
instrucciones constituyentes sobre la máquina abstracta, de acuerdo a la
semántica operacional que se define iterativamente en cada capítulo.

Este enfoque es sumamente efectivo dado que le permite al alumno comprender el
mecanismo de abstracción que fundamenta el aprendizaje de la programación: el
lenguaje de programación es una interfase con una máquina subyacente, y su
diseño está a su vez estratificado en niveles de abstracción sobre una
semántica base (procedimientos, funciones, functores, objetos y otros). Aún
más, este conocimiento es entendido sin necesidad de comprender la operativa
física de la máquina subyacente real, lo que requiere contar con experiencia en
ciencias básicas (física, química) y aplicadas (electrónica).

Por último, Oz expone una variedad de paradigmas que permite al docente diseñar
un currículo que incorpore conceptos básicos en cada uno de ellos. Esto es
especialmente importante hoy en día, donde las arquitecturas basadas en
microservicios buscan explotar la separación de los componentes de un sistema
informático en microsistemas desarrollados con diferentes lenguajes de
programación de manera de aplicar paradigmas y técnicas más efectivas en cada
dominio.

\subsection{Herramientas existentes}

Actualmente existe un único entorno de desarrollo, administrado y mantenido por
The Mozart Board. El entorno es extremadamente completo, pero tiene una gran
debilidad: debido a que la misión fundacional de The Mozart Board es expandir
el campo de aplicabilidad del lenguaje Oz más allá del nicho de la enseñanza,
suele ser complejo para ser abordado por alguien que recién se está iniciando
en el campo de la programación.

En particular, podemos destacar las siguientes dificultades asociadas al uso de
este entorno en el marco educativo:

\begin{enumerate}

\item El sistema está basado en GNU Emacs. Si bien este constituye un excelente
  entorno de desarrollo en general, tiene una curva de aprendizaje desfavorable y
  requiere una inversión de tiempo considerable para poder operarlo.

\item Muchos de los conceptos subyacentes a la máquina de ejecución abstracta
  están escondidos dentro del entorno de ejecución, y el entorno de desarrollo
  no provee herramientas para acceder a esta información.

\item El sistema es una aplicación binaria de escritorio, lo que significa que
  requiere un proceso de instalación que constituye nuevamente una barrera
  adicional para el neófito, sobre todo si se tiene en cuenta que no existen
  paquetes binarios actualizados para las diferentes plataformas que puede
  tener el alumno.

\item El sistema no puede ser ejecutado sobre un dispositivo móvil, por lo que
  requiere que el alumno cuente con una computadora para poder ejecutar,
  evaluar e interactuar con el software, cuando a priori podría utilizar
  cualquiera de los dispositivos móviles que suelen estar más al alcance de la
  mano.

\end{enumerate}

\subsection{Propuesta}

En base a lo descripto anteriormente, se propone implementar un nuevo entorno
de desarrollo cuya misión sea facilitar la comprensión de las características
operativas del entorno de ejecución de Oz.

\section{Alcance}

\section{Herramientas de desarrollo}

\section{Metodología}

\section{Cronograma}

\section{Curriculum Vitae}

\section{Plan de cursada}

\end{document}

