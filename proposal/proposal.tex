\documentclass[a4paper,11pt]{article}

%%%%%%%%%%%%%%%%%%%%%%%%%%%%%%%%%%%%%%%%%%%%%%%%%%%%%%%%%%%%%%%%%%%%%%%%
% Paquetes utilizados
%%%%%%%%%%%%%%%%%%%%%%%%%%%%%%%%%%%%%%%%%%%%%%%%%%%%%%%%%%%%%%%%%%%%%%%%
% Soporte para el lenguaje español
\usepackage{textcomp}
\usepackage[utf8]{inputenc}
\usepackage[T1]{fontenc}
\usepackage[spanish]{babel}

% Configuración de página
\usepackage[lmargin=1in, rmargin=1in]{geometry}
\setlength{\parskip}{0.75em}
\setlength{\parindent}{1em}

% Graficos
\usepackage{graphicx}

% Encabezados y pies de pagina
\usepackage{fancyhdr}
\setlength{\headheight}{15.2pt}
\pagestyle{fancy}
\fancyhf{}
\lhead{Kozily}
\rhead{Propuesta de Proyecto}
\cfoot{\thepage}

% Tablas
\usepackage{longtable}
\usepackage{array}
\newcolumntype{L}[1]{>{\raggedright\let\newline\\\arraybackslash\hspace{0pt}}m{#1}}

% Urls
\usepackage[hidelinks]{hyperref}

% Referencias
\usepackage[nottoc,numbib]{tocbibind}

\begin{document}

%%%%%%%%%%%%%%%%%%%%%%%%%%%%%%%%%%%%%%%%%%%%%%%%%%%%%%%%%%%%%%%%%%%%%%%%
% Titulo
%%%%%%%%%%%%%%%%%%%%%%%%%%%%%%%%%%%%%%%%%%%%%%%%%%%%%%%%%%%%%%%%%%%%%%%%

\thispagestyle{empty}

\begin{titlepage}

  \newcommand{\HRule}{\rule{\linewidth}{0.5mm}}
  \newenvironment{bottompar}{\par\vspace*{\fill}}{\clearpage}

  \center

  \textsc{\LARGE Universidad de Buenos Aires}\\[0.5cm]
  \textsc{\Large Facultad de Ingeniería}\\[1.5cm]

  \includegraphics[scale=0.5]{assets/logo.png}\\[1cm]


  \textsc{\large (75.99) Trabajo Profesional}\\[0.25cm]
  {\huge \bfseries Propuesta de Proyecto} \\[0.4cm]
  \HRule \\[0.4cm]
  {\huge \bfseries Kozily} \\[0.4cm]
  Un entorno de desarrollo de Oz para estudiantes
  \HRule

  \begin{bottompar}
    \begin{minipage}[t]{.45\linewidth}
      \begin{flushleft}
        {\bfseries Autores:}

        Arana, Andrés          - P. 86.203

        Piano, Sergio          - P. 85.191
      \end{flushleft}
    \end{minipage}
    \hfill
    \begin{minipage}[t]{.45\linewidth}
      \begin{flushright}
        {\bfseries Tutor:}

        Rosa Wachenchauzer
      \end{flushright}
    \end{minipage}
  \end{bottompar}

\end{titlepage}

%%%%%%%%%%%%%%%%%%%%%%%%%%%%%%%%%%%%%%%%%%%%%%%%%%%%%%%%%%%%%%%%%%%%%%%%
% Documento
%%%%%%%%%%%%%%%%%%%%%%%%%%%%%%%%%%%%%%%%%%%%%%%%%%%%%%%%%%%%%%%%%%%%%%%%

%-----------------------------------------------------------------------
% Abstract
%-----------------------------------------------------------------------
\begin{abstract}

  El presente documento constituye una propuesta de proyecto enmarcado en el
  contexto de un trabajo profesional para la carrera de Ingeniería en
  Informática en la Facultad de Ingeniería de la Universidad de Buenos Aires.
  Se propone desarrollar un entorno de desarrollo para el lenguaje de
  programación Oz orientado a apoyar el aprendizaje de los diferentes
  paradigmas que soporta el mismo, y que constituye su principal caso de uso.

\end{abstract}
\clearpage

%-----------------------------------------------------------------------
% Tabla de contenidos
%-----------------------------------------------------------------------
\tableofcontents
\clearpage

%-----------------------------------------------------------------------
% Objetivo
%-----------------------------------------------------------------------
\section{Objetivo}

\subsection{Introducción}

Oz \cite{oz} es un lenguaje multiparadigma diseñado por Gert Smolka en la
Université Catholique de Louvain en 1991, específicamente pensado para enseñar
conceptos de programación a alumnos universitarios. Fue desarrollado
posteriormente en 1996 por Peter Van-Roy y Seif Haridi de Swedish Institute of
Computer Science en conjunto con un manual de estudio canónico: Concepts,
Techniques and Models of Computer Programming \cite{ctmcp}. Finalmente, en 2005
se constituyó una entidad propia, The Mozart Board \cite{mozboard}, para
coordinar los esfuerzos de desarrollo del lenguaje, su especificación y entorno
de ejecución.

Si bien la visión fundacional de The Mozart Board es ampliar el espectro de
aplicabilidad del lenguaje, en la actualidad la mayoría de los casos de uso del
mismo se nuclean alrededor de la idea original que sustenta su diseño: como
herramienta para enseñar a alumnos universitarios diversos conceptos, técnicas
y modelos de programación.

En Argentina, tanto la Facultad de Ingeniería de la Universidad de Buenos Aires
(curso 75.31 para Ingeniería en Informática e Ingeniería en Electrónica, cursos
75.24 y 95.07 para Licenciatura en Análisis de Sistemas) como en la Facultad de
Matemática, Astronomía y Física de la Universidad Nacional de Córdoba enmarcan
la enseñanza de paradigmas y lenguajes de programación en el enfoque de
Concepts, Techniques and Models of Computer Programming, y por consiguiente,
usando Oz como lenguaje de estudio.

\subsection{Concepts, Techniques and Models of Computer Programming}

El manual de referencia de Peter Van-Roy y Seif Haridi \cite{ctmcp} enmarca una
metodología incremental de enseñanza de los diferentes paradigmas, conceptos y
técnicas de programación.  En el mismo, el autor desarrolla un marco teórico de
ejecución, la \emph{máquina abstracta de ejecución}, un sistema matemático
abstracto que define la ejecución de un programa en base a un conjunto
minimalista de conceptos: un stack de ejecución con diferentes instrucciones,
un entorno de ejecución que asigna identificadores a celdas de memoria y una
memoria que almacena valores en celdas. Las diferentes técnicas, conceptos y
modelos que son examinados a lo largo del libro se definen en base a los
efectos de las instrucciones constituyentes sobre la máquina abstracta, de
acuerdo a la semántica operacional que se define iterativamente en cada
capítulo.

Este enfoque es sumamente efectivo dado que le permite al alumno comprender el
mecanismo de abstracción que fundamenta el aprendizaje de la programación: el
lenguaje de programación es una interfase con una máquina subyacente, y su
diseño está a su vez estratificado en niveles de abstracción sobre una
semántica base (procedimientos, funciones, functores, objetos y otros). Aún
más, este conocimiento es entendido sin necesidad de comprender la operativa
física de la máquina subyacente real, lo que requiere contar con experiencia en
ciencias básicas (física, química) y aplicadas (electrónica).

Por último, Oz \cite{oz} expone una variedad de paradigmas que permite al
docente diseñar un currículo que incorpore conceptos básicos en cada uno de
ellos. Esto es especialmente importante hoy en día, donde las arquitecturas
basadas en microservicios buscan explotar la separación de los componentes de
un sistema informático en microsistemas desarrollados con diferentes lenguajes
de programación de manera de aplicar paradigmas y técnicas más efectivas en
cada dominio.

\subsection{Herramientas existentes}

Actualmente existe un único entorno de desarrollo para el lenguaje denominado
Mozart \cite{mozart}, administrado y mantenido por The Mozart Board
\cite{mozboard}. El entorno es extremadamente completo, pero tiene una gran
debilidad: debido a que la misión fundacional de The Mozart Board es expandir
el campo de aplicabilidad del lenguaje Oz más allá del nicho de la enseñanza,
suele ser complejo para ser abordado por alguien que recién se está iniciando
en el campo de la programación.

En particular, podemos destacar las siguientes dificultades asociadas al uso de
este entorno en el marco educativo:

\begin{enumerate}

  \item El sistema está basado en GNU Emacs \cite{emacs}. Si bien este
    constituye un excelente entorno de desarrollo en general, tiene una curva
    de aprendizaje desfavorable y requiere una inversión de tiempo considerable
    para poder operarlo.

  \item Muchos de los conceptos subyacentes a la máquina de ejecución abstracta
    están escondidos dentro del entorno de ejecución, y el entorno de desarrollo
    no provee herramientas para acceder a esta información.

  \item El sistema es una aplicación binaria de escritorio, lo que significa que
    requiere un proceso de instalación que constituye nuevamente una barrera
    adicional para el neófito, sobre todo si se tiene en cuenta que no existen
    paquetes binarios actualizados para las diferentes plataformas que puede
    tener el alumno.

  \item El sistema no puede ser ejecutado sobre un dispositivo móvil, por lo que
    requiere que el alumno cuente con una computadora para poder ejecutar,
    evaluar e interactuar con el software, cuando a priori podría utilizar
    cualquiera de los dispositivos móviles que suelen estar más al alcance de la
    mano.

\end{enumerate}

\subsection{Propuesta}

En base a lo descripto anteriormente, se propone implementar un nuevo entorno
de desarrollo cuya misión sea facilitar la comprensión de las características
operativas del entorno de ejecución de Oz.

%-----------------------------------------------------------------------
% Alcance
%-----------------------------------------------------------------------
\section{Alcance}

\subsection{Requerimientos funcionales}

Los requerimientos funcionales del proyecto se dividen en dos categorías. Por
un lado, aquellos requerimientos relacionados a la interpretación de
construcciones sintácticas y semánticas que constituyen los diferentes
paradigmas soportados por Oz. Por el otro, los asociados a las funcionalidades
didácticas a través de las que el estudiante interactúa con el entorno de
ejecución.

\subsubsection{Construcciones semánticas}

El sistema deberá ejecutar correctamente todas las construcciones sintácticas y
semánticas de los siguientes paradigmas y modelos , de acuerdo a su definición
en Concepts, Techniques and Models of Computer Programming:

\begin{enumerate}

  \item Modelo declarativo \cite[cap.2]{ctmcp}

  \item Técnicas aplicadas del modelo declarativo \cite[cap.3]{ctmcp}

  \item Concurrencia declarativa \cite[cap.4]{ctmcp}

  \item Concurrencia basada en pasaje de mensajes \cite[cap.5]{ctmcp}

  \item Estado explícito \cite[cap.6]{ctmcp}

  \item Programación orientada a objetos \cite[cap.7]{ctmcp}

  \item Concurrencia de estado compartido \cite[cap.8]{ctmcp}

  \item Programación relacional \cite[cap.9]{ctmcp}

  \item Programación de restricciones \cite[cap.12]{ctmcp}

\end{enumerate}

\subsubsection{Funcionalidades didácticas}

El sistema deberá permitir que el alumno interactúe con el entorno de ejecución
de las siguientes maneras:

\begin{enumerate}

  \item El sistema deberá permitir al alumno ingresar un programa en un
    editor de textos embebido.

  \item El sistema deberá marcar con diferentes estilos de fuentes y/o
    colores las diferentes construcciones sintácticas y semánticas del
      lenguaje a medida que el alumno ingresa un programa en el editor.

    \item El sistema deberá verificar la sintaxis del programa a medida que el
      alumno lo ingresa en el editor, reportando cualquier error que detecte.

    \item El sistema deberá ejecutar el programa ingresado, recopilando las
      expresiones de inspección de tipo \textit{Browse} y mostrando estas
      salidas recopiladas.

    \item El sistema deberá mostrar la traducción del programa ingresado a
      lenguaje kernel.

    \item El sistema deberá mostrar los pasos de ejecución del programa
      ingresado, haciendo explícita la máquina abstracta de ejecución.

\end{enumerate}

\subsubsection{Funcionalidades adicionales}

Las siguientes son funcionalidades adicionales, que constituyen un backlog de
requerimientos de tipo nice to have, y serán implementados de acuerdo a la
dinámica iterativa de la planificación:

\begin{enumerate}

  \item El sistema deberá permitir la registración de un alumno a través de
    proveedores de mecanismos de SSO como Google \cite{googleoauth}, Facebook
    \cite{facebookoauth} o Github \cite{githuboauth}.

  \item El sistema deberá permitir al alumno guardar y cargar los programas
    que ingrese en el mismo.

  \item El sistema deberá permitir al alumno compartir programas ingresados
    con otros alumnos, permitiendo a estos cargar programas ingresados.

  \item El sistema deberá permitir que más de un alumno edite un mismo
    programa al mismo tiempo, de una manera colaborativa.

\end{enumerate}

\subsection{Requerimientos no funcionales}

\subsubsection{Hardware y software}

El sistema deberá ser distribuido como una aplicación web, y debe poder usarse
incluso si la conectividad es limitada o nula de acuerdo a los lineamientos de
una aplicación web progresiva \cite{pwa}. La aplicación deberá poder ejecutarse
en cualquier plataforma, tanto móvil como tradicional, en la que pueda
ejecutarse google chrome \cite{chrome}.

De acuerdo a los lineamientos de este software, los requerimientos mínimos de
acuerdo a la plataforma son los siguientes:

\begin{enumerate}

  \item \textbf{Windows} Windows 7 o superior, con un procesador Intel Pentium
    4 o superior con soporte de SSE2 y al menos 2GB de memoria RAM.

  \item \textbf{Mac} OS X Maverick 10.9 o superior.

  \item \textbf{Linux} Ubuntu 14.04+, Debian 8+, openSUSE 13.1+ o Fedora Linux
    21+, con un procesador Intel Pentium 4 o superior con soporte de SSE2 y al
    menos 2GB de memoria RAM.

  \item \textbf{Teléfonos o Tablets Android} Cualquier dispositivo con Android
    4.1 o superior que soporte la instalación de google chrome.

  \item \textbf{iPhone y iPad} Cualquier dispositivo con iOS9 o superior que
    soporte la instalación de Google Chrome.

\end{enumerate}

\subsubsection{Idioma}

La aplicación, así como toda la documentación de proyecto y de producto será
desarrollada enteramente y únicamente en inglés, para fomentar la colaboración
con las universidades relacionadas con el lenguaje.

%-----------------------------------------------------------------------
% Herramientas de desarrollo
%-----------------------------------------------------------------------
\section{Herramientas de desarrollo}

\subsection{Hardware}

Cada desarrollador cuenta con una computadora personal, Asus UX303LN y Apple
MacBook Pro. Adicionalmente, para las pruebas en dispositivos móviles se
utilizaran los celulares OnePlus2, Moto G 2nd Gen y iPhone 6s, las tablets
Samsung Galaxy Tab S2 y iPad.

\subsection{Software}

Las principales herramientas de software sobre las que se apoya el desarrollo
del sistema son las siguientes.

\begin{description}

  \item[NodeJS] \textit{https://nodejs.org/} Lenguaje de programación
    funcional, escrito en javascript, se maneja a través de eventos y trabaja con
    un modelo de entrada y salida no bloqueante, que le permite ser un lenguaje
    liviano y eficiente. Open Source.

  \item[NPM] \textit{https://www.npmjs.com/} Manejador de paquetes de nodeJS,
    es el ecosistema más grande en dónde se encuentran las librerías open Source
    de node.

  \item[VIM] \textit{http://www.vim.org/} Editor de texto. Open Source.

  \item[Git] \textit{https://git-scm.com/} Software de versionado de código.
    Open Source.

  \item[GitHub] \textit{https://github.com/} Software de gestión de
    repositorios de código, incluyendo seguimiento de tareas, hitos y
    entregables.

  \item[Waffle] \textit{https://waffle.io} Plataforma de gestión de proyectos
    integrado con GitHub.

  \item[TravisCI] \textit{https://travis-ci.org/} Software de integración y
    entrega continua.

  \item[AWS] \textit{https://aws.amazon.com/} Infraestructura a demanda
    y cloud computing para necesidades de hosting.

  \item[LaTeX] \textit{https://www.latex-project.org/} Generación de
    documentación.  Open Source.

  \item[Toggl] \textit{https://toggl.com/} Registración y reporte de horas de
    trabajo.

  \item[Docker] \textit{https://www.docker.com/} Plataforma de virtualización
    para entornos productivos y de desarrollo. Open Source.

  \item[React] \textit{https://facebook.github.io/react/} Librería para
    interfaces reactivas en la web. Open Source.

\end{description}

%-----------------------------------------------------------------------
% Metodología
%-----------------------------------------------------------------------
\section{Metodología}

Se adoptará une metodología iterativa basada en Kanban \cite{kanban} y Scrum
\cite{scrum} y adaptada al desarrollo de software, en conjunto con una
planificación de alto nivel de hitos intermedios a cumplir cada mes.

Todas las tareas se detallarán como issues en GitHub utilizando un template de
User Story \cite{userstory} y se visualizarán como un Kanban Board
\cite{kanbanboard} a través de Waffle. El Kanban board tendrá los siguientes
estados:

\begin{description}

  \item[Backlog] Contiene todas las tareas que aún no se han desarrollado. Las
    tareas en este estado estarán priorizadas de acuerdo a los entregables de
    los siguientes hitos y no serán asignadas a priori a ninguno de los
    miembros del equipo.

  \item[Ready] Contiene todas las tareas del próximo hito.

  \item[In Progress] Contiene todas las tareas que están en desarrollo. Estas
    están asignadas a un miembro del equipo obligatoriamente, responsable de
    hacer el seguimiento hasta que esté lista.

  \item[Done] Contiene todas las tareas que fueron completadas. Para que una
    tarea esté en este estado debe estar integrada al master.

\end{description}

Cada tarea será tomada por un miembro del equipo del estado de \textit{Ready},
será asignada al miembro que corresponda será puesta en \textit{In Progress}.
Una vez que el desarrollador considera que la misma está lista, habiendo
implementado los cambios necesarios y los tests automatizados que verifiquen la
implementación, sube los cambios a un branch propio para la funcionalidad en
GitHub y crea un pedido de integración. El sistema de integración continua toma
los cambios del branch y ejecuta todos los tests, adjuntando al pedido de
integración el informe de pruebas. Con toda esta información, otro miembro del
equipo revisa el pedido, realiza los comentarios pertinentes y una vez que está
satisfecho con los cambios integra los cambios al master. En este punto, el
sistema de integración continua ejecuta nuevamente las pruebas y realiza la
implantación en el ambiente productivo de forma automática. Una vez que los
cambios están disponibles en el ambiente productivo, el desarrollador
mueve la tarea al estado \textit{Done}.

Una vez por semana, los días domingo o lunes se realizará una reunión de estado
dentro el equipo similar a la reunión de scrum \cite{standup} de la
metodología homónima.

Cada dos semanas se realizará una reunión para revisar el estado de las tareas
del hito correspondiente, ajustar la planificación según corresponda y hacer
una retrospectiva \cite{retro} que permita revisar los parámetros y prácticas
de calidad del proyecto. Los resultados de la reunión se sumarizarán en un
informe de avance de proyecto que será enviado al tutor.

%-----------------------------------------------------------------------
% Cronograma
%-----------------------------------------------------------------------
\section{Cronograma}

Se planifica una carga de trabajo por persona por semana de un mínimo de 15
horas, con iteraciones de dos semanas e hitos entregables mensuales. Bajo este
esquema, se detalla a continuación un cronograma de alto nivel del proyecto:

\begin{longtable}{clL{0.5\textwidth}}
  \hline
  \textbf{Fecha} & \textbf{Evento} & \textbf{Notas} \\
  \hline
  \endhead

  \hline
  \multicolumn{3}{r}{\textit{Continua en la próxima página}} \\
  \endfoot

  \hline
  \endlastfoot

  15/08/2016 - 28/08/2016 & Iteración 1 & 60 horas \\

  \hline 29/08/2016 & Hito 1 & Entregables: Entornos, UI básica, ingreso de código fuente \\ \hline

  29/08/2016 - 11/09/2016 & Iteración 2 & 60 horas \\
  12/09/2016 - 25/09/2016 & Iteración 3 & 60 horas \\

  \hline 26/09/2016 & Hito 2 & Entregables: Modelo declarativo, marcado de sintaxis, visualización de ejecución \\ \hline

  26/09/2016 - 09/10/2016 & Iteración 4 & 60 horas \\
  10/10/2016 - 23/10/2016 & Iteración 5 & 60 horas \\

  \hline 24/10/2016 & Hito 3 & Entregables: Técnicas de modelo declarativo, visualización de kernel y máquina abstracta \\ \hline

  24/10/2016 - 06/11/2016 & Iteración 6 & 60 horas \\
  07/11/2016 - 20/11/2016 & Iteración 7 & 60 horas \\

  \hline 21/11/2016 & Hito 4 & Entregables: Concurrencia declarativa \\ \hline

  21/11/2016 - 04/12/2016 & Iteración 8 & 60 horas \\
  05/12/2016 - 18/12/2016 & Iteración 9 & 60 horas \\

  \hline 19/12/2016 & Hito 5 & Entregables: Estado explícito y programación orientada a objetos \\ \hline

  19/12/2016 - 01/01/2017 & Iteración 10 & 60 horas \\
  02/01/2017 - 15/01/2017 & Iteración 11 & 60 horas \\

  \hline 16/01/2017 & Hito 6 & Entregables: Concurrencia de estado compartido \\ \hline

  16/01/2017 - 29/01/2017 & Iteración 12 & 60 horas \\
  30/01/2017 - 12/02/2017 & Iteración 13 & 60 horas \\

  \hline 13/02/2017 & Hito 7 & Entregables: Programación relacional, programación de restricciones \\ \hline

  13/02/2017 - 19/02/2017 & Iteración 14 & 30 horas \\

  \hline 19/02/2017 & Hito 8 & Cierre de proyecto \\ \hline

  \textbf{Total de horas} & & 810 horas
\end{longtable}

%-----------------------------------------------------------------------
% Curriculum
%-----------------------------------------------------------------------
\section{Currículum Vítae}

\subsection{Andrés Gastón Arana}

\subsubsection{Datos personales}

\noindent \begin{tabular}{l l l l}
  \textbf{Nombre} & Andrés Gastón Arana & \textbf{Teléfono} & +54 9 11 3162-7877\\
  \textbf{Padrón} & 86.203              & \textbf{Email}    & and2arana@gmail.com \\
\end{tabular}

\subsubsection{Educación}

\noindent \textbf{Ingeniería en Informática}

\noindent\emph{2004 - Actual}

\noindent Cursando Ingeniería en Informática en la Facultad de Ingeniería de la
Universidad de Buenos Aires. \\

\noindent \textbf{Polimodal con orientación en adm. y gestión de empresas}

\noindent\emph{2000 - 2003}

\noindent Bachiller con orientación en adm. y gestión de empresas en el colegio
Mariano Moreno.

\subsubsection{Experiencia laboral}

\noindent \textbf{Socio fundador en Recompensa.mobi}

\noindent \emph{2014 - Actual}

\noindent Participo actualmente en el desarrollo, comercialización e
implementación de una plataforma de registro de eventos asociados con los
procesos de ventas y customer service en PyMes. Desarrollo diversos roles,
principalmente relacionados al relevamiento de requerimientos, diseño,
conceptualización de arquitectura e implementación. \\

\noindent \textbf{Desarrollador en Global Fishing Watch}

\noindent \emph{2015 - Actual}

\noindent Desempeño tareas de consultoría como desarrollador en Global Fishing
Watch, una ONG que realiza monitoreo satelital de barcos de pesca. Involucrado
en el procesamiento de eventos AIS para detectar patrones de pesca, que incluye
técnicas probabilísticas, heurísticas y de machine learning aplicados a
100.000.000 de registros por día, así como también infraestructura de
visualización de la información generada. \\

\noindent \textbf{Líder técnico en Asante}

\noindent \emph{2011 - 2014}

\noindent Tareas de líder técnico para diversos proyectos en Asante. Trabajé
principalmente coordinando un equipo de 5 desarrolladores y realizando tareas
de desarrollo para varios proyectos que involucraban la generación de
aplicaciones para dispositivos móviles Android, iOS y web nativo, incluyendo un
backend administrativo en Ruby. \\

\noindent \textbf{Líder técnico en PlayPhone}

\noindent \emph{2011 - 2011}

\noindent Desarrollé tareas de líder técnico en PlayPhone, trabajando
principalmente en la coordinación e implementación de sistemas de backend de
una plataforma de servicios para aplicaciones móviles como autorización,
gestión de pagos y microtransacciones en .NET. \\

\noindent \textbf{Desarrollador en LeaseTrader}

\noindent \emph{2010 - 2011}

\noindent Tareas de desarrollo bajo la plataforma .NET para un sitio de
compra-venta de leases de autos en estados unidos, incluyendo principalmente el
desarrollo y testing de nuevos features, así como el diseño, arquitectura e
implementación de nuevas plataformas de soporte de la compañía. \\

\noindent \textbf{Desarrollador en Sabarasa Entertainment}

\noindent \emph{2009 - 2010}

\noindent Desarrollo de videojuegos para consolas Wii en lenguajes de bajo
nivel, principalmente C++, incluyendo tareas de modelado matemático y
optimización de bajo nivel para la plataforma, incluyendo sistemas de
procesamiento en tiempo real de entrada y subsistemas gráficos. \\

\noindent \textbf{Desarrollador en Globant}

\noindent \emph{2008 - 2009}

\noindent Tareas de desarrollo en múltiples clientes, pero principalmente en un
proyecto de modernización para una plataforma de servicios y distribución de
contenido en sistemas móviles que incluía la reingeniería de un sistema de
facturación y autorización en un stack JVM, Spring y JBoss. \\

\noindent \textbf{Desarrollador en EDS}

\noindent \emph{2007 - 2008}

\noindent Tareas de desarrollo en un equipo chico a cargo de la modernización
de la infraestructura de procesamiento de aprobación de créditos para la compra
de autos en GMAC. El proyecto consistía en la migración de una infraestructura
de más de 10 años desarrollada en C++ a una plataforma .NET. \\

\noindent \textbf{Desarrollador en LanTec IT}

\noindent \emph{2007 - 2008}

\noindent Tareas de desarrollo en el proyecto de QuickPass, un emprendimiento de procesamiento biométrico para control de acceso y registro de horas laborales. Componentes desarrollados en C++ para interactuar con el hardware, y en .NET para el resto de las aplicaciones. Desempeñé también tareas de electrónica para construir dispositivos de toma de huellas dactilares en base a componentes importados. \\

\noindent \textbf{Desarrollador en Sistemas Bejerman}

\noindent \emph{2004 - 2007}

\noindent Tareas de desarrollo los módulos de contabilidad, balances, recursos
humanos y sueldos y jornales del ERP que constituía el producto principal de la
empresa, una plataforma .NET. Adicionalmente codificaba los instaladores del
sistema, bajo la tecnología de instalación MSI, con binarios precompilados en
Delphi. \\

\subsubsection{Habilidades}

\begin{enumerate}

  \item Experiencia en trabajo en equipos de desarrollo bajo diversas
    metodologías, tanto ágiles (Kanban, Scrum) como tradicionales.

  \item Experiencia en trabajo bajo condiciones de continuous integration y
    continuous delivery.

  \item Conocimiento avanzado en diversos lenguajes de programación. C++,
    Java, C\#, F\#, Ruby, Python, Clojure, LISP, Haskell y Javascript.

  \item Extensiva experiencia laboral en sistemas distribuidos web, incluyendo
    arquitecturas REST, microservicios sobre HTTP y aplicaciones web.

  \item Experiencia laboral en desarrollo de clientes progresivos sobre
    exploradores móviles y de escritorio, con aplicación de tecnologías como
    HTML5, CSS3, responsive design, progressive enhancement, service workers y
    offline first clients.

  \item Experiencia laboral sobre plataformas y librerías de desarrollo para
    cada entorno, como ser Spring, JUnit, JMock, JBoss para plataformas Java,
    Rails, RSpec, Cucumber para plataformas Ruby, Django para python, Compojure y Ring para
    Clojure, etc..

  \item Experiencia laboral sobre data modeling y data stores, tanto SQL
    (MySQL, Oracle, MSSQL, Postgresql) como noSQL (MongoDB, Google Cloud
    Datastore, Redis).

  \item Experiencia laboral sobre BigData y las tecnologías relacionadas, como
    Spark, Hadoop y Kafka. Incluye aplicación de técnicas de análisis
    heurísticos, probabilísticos y de machine learning (modelos bayesianos y
    redes neuronales).

  \item Experiencia laboral en diversas plataformas de cloud computing, como
    AWS y Google Cloud Platform. Entornos virtualizados y en base a
      contenedores (Docker) con coordinadores como Kubernetes.

\end{enumerate}

\subsubsection{Idiomas}

\begin{enumerate}

  \item Español nativo.

  \item Inglés avanzado. Nivel B en CAE, A en FCE.

  \item Francés principiante. Sin certificación avalada.

\end{enumerate}

\subsubsection{Materias aprobadas}

\begin{longtable}{|l|l|r|l|}
  \hline
  Materia                                       & Fecha      & Nota & Acta              \\
  \hline
  \endhead

  \hline
  \multicolumn{4}{r}{\textit{Continua en la próxima página}} \\
  \endfoot

  \hline
  \endlastfoot

  (6201) Física I A                          & 26/07/2005 & 6  & 2-104-20          \\
  (7540) Algoritmos Y Programación I         & 20/12/2005 & 7  & 17-93-191         \\
  (6108) Álgebra II A                        & 21/12/2005 & 5  & 1-150-126         \\
  (6103) Análisis Matemático II A            & 25/07/2006 & 4  & 1-151-98          \\
  (7541) Algoritmos Y Programación II        & 14/08/2006 & 8  & 17-95-50          \\
  (6301) Química                             & 17/08/2006 & 6  & 3-73-44           \\
  (7507) Algoritmos Y Programación III       & 14/12/2006 & 8  & 17-95-126         \\
  (6203) Física II A                         & 21/02/2008 & 7  & 2-106-76          \\
  (6670) Estructura Del Computador           & 12/08/2008 & 6  & 6-136-98          \\
  (7512) Análisis Numérico I                 & 28/12/2009 & 5  & 17-102-191        \\
  (7506) Organización De Datos               & 21/02/2011 & 8  & 17-106-24         \\
  (7542) Taller De Programación I            & 01/03/2011 & 9  & 17-106-69         \\
  (6109) Probabilidad Y Estadística B        & 21/07/2011 & 5  & 1-159-64          \\
  (6602) Laboratorio                         & 11/08/2011 & 8  & 6-141-34          \\
  (7508) Sistemas Operativos                 & 22/12/2011 & 9  & 17-108-99         \\
  (7112) Estructura De Las Organizaciones    & 08/02/2012 & 7  & 11-153-160        \\
  (7509) Análisis De La Información          & 07/08/2012 & 6  & 17-110-1          \\
  (6110) Análisis Matemático III A           & 09/08/2012 & 6  & 1-157-164         \\
  (7114) Modelos Y Optimización I            & 12/12/2012 & 8  & 11-155-28         \\
  (6620) Organización De Computadoras        & 04/02/2013 & 7  & 6-143-56          \\
  (7514) Lenguajes Formales                  & 25/02/2013 & 9  & 17-111-129        \\
  (7113) Información En Las Organizaciones   & 16/07/2013 & 4  & 71-0001053        \\
  (7526) Simulación                          & 31/07/2013 & 7  & 95-0001278        \\
  (7510) Técnicas De Diseño                  & 05/08/2013 & 7  & 95-0001318        \\
  (7545) Taller De Desarrollo De Proyectos I & 12/12/2013 & 8  & 95-0001636        \\
  (7515) Base De Datos                       & 26/02/2014 & 5  & 95-0001833        \\
  (6671) Sistemas Gráficos                   & 18/07/2014 & 6  & 86-0001914        \\
  (7552) Taller De Programación II           & 08/08/2014 & 5  & 95-0002323        \\
  (7559) Técnicas De Prog. Concurrente I     & 12/12/2014 & 7  & 95-0002481        \\
  (7544) Adm. Y Control De Proy. Inf. I      & 16/12/2014 & 4  & 95-0002510        \\
  (7547) Taller De Desarrollo De Proy. II    & 29/06/2015 & 8  & 95-0002895        \\
  (7140) Leg. Y Ej. Prof. De La Ing. En Inf. & 30/07/2015 & 5  & 71-0002132        \\
  (7531) Teoría De Lenguaje                  & 14/12/2015 & 10 & 95-0003624        \\
  (7550) Introducción A Los Sist. Int.       & 29/02/2016 & 7  & 95-0003649        \\
  \hline
\end{longtable}

\textbf{Promedio académico}: 6,68

\subsubsection{Plan de cursada}

\textbf{Primer cuatrimestre 2016}

\begin{enumerate}
  \item (7548) Calidad en el Desarrollo de Sistemas
  \item (6215) Física III D
  \item (7546) Admin. y Gestión de Proy. Informáticos II
\end{enumerate}

\textbf{Primer cuatrimestre 2017}

\begin{enumerate}
  \item (7543) Introducción a los Sistemas Distribuidos
  \item (7565) Manufactura Integrada por Comp. I
  \item (7567) Sist. Autom. de Diag. y Detección de Fallas I
\end{enumerate}

\subsection{Sergio Matías Piano}

\subsubsection{Datos personales}

\noindent \begin{tabular}{l l l l}
  \textbf{Nombre} & Sergio Matias Piano  & \textbf{Teléfono} & +54 11 6050-2400\\
  \textbf{Padrón} & 85.191               & \textbf{Email}    & smpiano@gmail.com \\
\end{tabular}

\subsubsection{Educación}

\noindent \textbf{Ingeniería en Informática}

\noindent\emph{2003 - Actual}

\noindent Cursando Ingeniería en Informática en la Facultad de Ingeniería de la
Universidad de Buenos Aires. \\

\noindent \textbf{Polimodal con orientación en adm. y gestión de empresas}

\noindent\emph{2000 - 2002}

\noindent Bachiller con orientación en administración y gestión de empresas en
el instituto Santo Cristo.

\subsubsection{Experiencia laboral}

\noindent \textbf{Desarrollador Ssr. de Moderación en MercadoLibre}

\noindent \emph{enero 2016 - presente}

\noindent Principales responsabilidades y tareas: Diseñador de la arquitectura
de nuevos proyectos en moderación. Desarrollador dedicado a las iniciativas
propuestas en moderación que impliquen modificar proyectos de equipos ajenos,
integrador. Tecnologías utilizadas: groovy, grails, js, nodejs, bash. \\

\noindent \textbf{Desarrollador Ssr. de Ventas en MercadoLibre}

\noindent \emph{enero 2015 - diciembre 2015}

\noindent Principales responsabilidades y tareas: Implementación y desarrollo
de tareas que involucran el área de ventas de mercadolibre.  Tecnologías
utilizadas: groovy, grails, js, nodejs, bash, Objective C, Android. \\

\noindent \textbf{Desarrollador Java Sr en OutSideIQ para BVision}

\noindent \emph{octubre 2014 - diciembre 2014}

\noindent Principales responsabilidades y tareas: Construcción de la
herramienta de OIQ para web-crawling con objetivo de reconstruir información
sobre todas las secretarías de estado de los EEUU.  Tecnologías utilizadas:
RegEx, java. \\

\noindent \textbf{Desarrollador Java Sr en NetSuite – Integración con D\&B
Bussiness information (Continuación) para BVision}

\noindent \emph{agosto 2014 - octubre 2014}

\noindent Principales responsabilidades y tareas: Integrar nuevos servicios de
D\&B con los de NetSuite para acceder a información más específica de cada
compañía.  Tecnologías utilizadas: NetSuite IDE, Javascript, NetSuite
framework, Web Services, REST, XAMP. \\

\noindent \textbf{Desarrollador Java Sr en Tv Publica – Envío de notificaciones
Push a dispositivos, para BVision}

\noindent \emph{junio 2014 - agosto 2014}

\noindent Principales responsabilidades y tareas: Evaluar framework de
AeroGear. Crear y modificar servicios exclusivos para la tv pública. Construir
librería para instalar servicio de push notifications en dispositivos Android y
iOS. Construir UI para administrar los mensajes a ser enviados. Testear flujo
completo en distintos dispositivos.  Tecnologías utilizadas: Jboss, Java,
Jax-Rs, Javascript, jQuery, Angular, AeroGear framework, Apache Cordova, Git.
\\

\noindent \textbf{Desarrollador Java Sr en NetSuite – Prueba de concepto,
integración con D\&B Bussiness information, para BVision}

\noindent \emph{abril 2014 - junio 2014}

\noindent Principales responsabilidades y tareas: Integrar los servicios de D\&B
con los de NetSuite para ubicar la compañía y la información relativa a ella de
manera tal que se completen automáticamente los formularios de carga.
Tecnologías utilizadas: NetSuite IDE, Javascript, NetSuite framework, Web
Services SOAP, REST. \\

\noindent \textbf{Desarrollador Java Sr en CollabNet - Teamforge para BVision}

\noindent \emph{julio 2013 - abril 2014}

\noindent Principales responsabilidades y tareas: Desarrollo de Teamforge.
Tecnologías utilizadas: Unix, Java, J2EE, Eclipse, SVN, Jboss, Web Services,
PostgreSQL, Angular, Jasmine, Selenium. \\

\noindent \textbf{Java Developer Sr. en CollabNet – CollabNet Connection
Framework, para BVision}

\noindent \emph{octubre 2012 - julio 2013}

\noindent Principales responsabilidades y tareas: Desarrollo, diseño y testeo
de servicios web para ser utilizados por el framework CCF con el fin de
sincronizar diferentes sistemas ALMs como por ejemplo, RequisitePro, TeamForge
y Rally.
Tecnologías utilizadas: Java, J2EE, Eclipse, SVN, Jboss, Web Services,
PostgreSQL, Agile methodology. \\

\noindent \textbf{Desarrollador Java Sr en Elastic Intelligence - Liberators,
para BVision}

\noindent \emph{febrero 2012 - octubre 2012}

\noindent Principales responsabilidades y tareas: Creación de un análisis API
por cliente para obtener información pre evaluada del mismo. Creación del
diseño a pedido del cliente y desarrollo de Zendesk Liberator y de Quickbase
Liberator. Contacto continuo con el cliente y respondiendo acorde a sus
necesidades.  Tecnologías utilizadas: Java, J2EE, Netbeans, SVN, Apache Tomcat,
Junit, Jettison, StAX, Web Services, PostgreSQL, Metodología Ágil. \\

\noindent \textbf{Desarrollador Senior y Líder de proyecto en Panera Bread para
Teracode}

\noindent \emph{octubre 2011 – febrero 2012}

\noindent Panera es una famosa cadena estadounidense de café, panadería y
restaurantes de comida rápida.  Se desarrolló una aplicación servidor y
aplicaciones clientes correspondientes.
El trabajo realizado fue exclusivamente de backend. Me fue encomendado liderar
el módulo de pagos, el cual involucra operaciones con tarjetas de crédito.
Lenguajes: Java, Groovy on Grails y Bash.
ORM: Ibatis.
Base de datos: Oracle.
Herramientas: Eclipse, Sqldeveloper, Tomcat, Gerrit.
Frameworks: REST, Spring, Spring Batch.
SCM: Git. \\

\noindent \textbf{Líder de proyecto en MOEO para Teracode}

\noindent \emph{junio 2011 –  octubre 2011}

\noindent MOEO es un juego de redes sociales que permite establecer
competencias entre amigos en base a los marcadores de partidos de baseball
verídicos en tiempo real.  El proyecto consistió en la construcción de un
servidor que brindara una API para el resto de las aplicaciones clientes,
iPhone, Android y Facebook.
Asumí el liderazgo del equipo de desarrollo y trabajé junto con los equipos
iPhone y Andriod.  Lenguajes: Java, Bash.
ORM: Hibernate
Base de datos: MySQL
Herramientas: Eclipse, Squirrel, Terracota, Tomcat, Maven, Gerrit.
SCM: Git. \\

\noindent \textbf{Desarrollador Senior en Citibank - Benefits para Teracode}

\noindent \emph{abril 2011 – junio 2011}

\noindent Citibank Benefits es la web de beneficios del banco Citi en la que se
puede navegar y explorar beneficios para los clientes.  Estuve a cargo,
principalmente, del desarrollo de una aplicación web que permitiera generar
nuevos beneficios.
Participé en el diseño, desarrollo, mantenimiento, análisis de requerimientos
funcionales y no funcionales y definición de arquitectura.  Lenguajes: Java.
Tecnologías: Spring, GWT, RestyGWT, Resteasy, GQuery, Lucene, Hibernate, Tomcat.
SCM: SVN. \\

\noindent \textbf{Desarrollador Senior en Citibank – Credit Cards Management
System, para Teracode}

\noindent \emph{abril 2011 –  junio 2011}

\noindent Se desarrolló una aplicación web para el Banco Citi con el propósito
de facilitar la solicitud de tarjetas de crédito, tanto para clientes de
distintas oficinas comerciales que hayan establecido un vínculo con el banco
como para automovilistas que circulan por estaciones de peaje.  Lenguajes:
Java.
Tecnologías: Maven, Spring, Hibernate, Wicket, iText, Css, MySQL, Tomcat.
SCM: SVN. \\

\noindent \textbf{Líder de proyecto en Ticket River para Teracode}

\noindent \emph{enero 2011 –  abril 2011}

\noindent Ticket River es un servicio que administra, promociona y busca
eventos recreativos. Además, provee un servicio de venta de tickets online y
offline por ventanilla.  Participé del desarrollo y mantenimiento de la
aplicación web, del análisis de requerimientos funcionales y no funcionales y
en la definición de la arquitectura.
Lenguajes: Java.
Tecnologías: Spring, Wicket, Liquidbase, Velocity, Maven, MySQL, PayPal and
Google Checkout.  SCM: SVN. \\

\noindent \textbf{Desarrollador Semi senior en Monitor Plus para Teracode}

\noindent \emph{junio 2009 –  enero 2011}

\noindent Desarrollo de una aplicación para el monitoreo y administración de
apuestas en carreras de caballos.  Colaboré en decisiones referidas al tipo de
arquitectura.
Asistí a los desarrolladores Juniors para que lograran una correcta integración
con el equipo de trabajo y las tecnologías utilizadas.  Participé en reuniones
con el cliente, aportando ideas y proponiendo soluciones sobre cómo debería
comportarse la aplicación.
Llevé a cabo presentaciones de los avances de la aplicación, en donde el
cliente podía manifestar su conformidad respecto del cumplimiento de los
requerimientos.  Código de alta calidad, asegurando una baja tasa de defectos y
aplicando Test Driven Development (TDD).
Lenguajes: Java.
Tecnologías: GWT, Spring, Hibernate, Rest-full, Apache Tomcat, Mysql,
Postgresql, Unix.  SCM: SVN y Git.
Metodologías: Agile y Waterfall. \\


\noindent \textbf{Consultora Novamens}

\noindent \emph{septiembre 2006 - mayo 2009}

\noindent \textbf{Desarrollador en Easy – Jumbo para Novamens}

\noindent Participé en el desarrollo de dos páginas web para la empresa Easy -
Jumbo.  Colaboré con el mantenimiento de la intranet sobre la sucursal de
Unicenter.
Tecnologías: jetty, Java, Python, Cocoon, Html, Xml, Xslt, Xsd, Xpath,
Javascript, Ajax, Flash, QT Design, Maven, Cvs, Sql, Postgresql. \\

\noindent \textbf{Desarrollador en The History Channel, A\&E, The Biography
Channel para Novamens}

\noindent Trabajé en el desarrollo completo de la página web “The History
Channel” para los usuarios de Latinoamérica. El desarrollo principal fue el del
calendario de programación del canal.  Además del desarrollo, tuve la
oportunidad de realizar documentación con UML (diagramas de clase y secuencia).
Tecnologías: Jetty, Java, Python, Cocoon, Wicket, Spring, Html, Xml, Xslt, Xsd,
Xpath, Javascript, Ajax, QT Design, Maven, Cvs, Sql, Postgresql. \\

\noindent \textbf{Desarrollador en Centralab-Fleni para Novamens}

\noindent Participé en el desarrollo de la intranet del laboratorio Fleni. Esta
web funcionaba como organizador, tanto de las historias clínicas de los
pacientes para un rápido acceso por parte de los médicos, como de los
experimentos llevados a cabo con la aplicación de drogas propias del
laboratorio. Participé además en el análisis e implementación de los
requerimientos.  Tecnologías: Jboss, Java, Cocoon, Jaas, Junit, Html, Xml,
Xslt, Xsd, Xpath, Javascript, Ajax, Ant, Monotone, Sql, Postgresql. \\

\noindent \textbf{Desarrollador en Lojack para Novamens}

\noindent Participé en el diseño e implementación de la página web de Lojack.
Trabajé en requerimientos funcionales y en el desarrollo.  Tecnologías: Jetty,
Java, Cocoon, Jaas, Wicket, Spring, Junit, Web Services, Wsdl, Axis, Html, Xml,
Xslt, Xsd, Xpath, Javascript, Ajax, Maven, Cvs, Sql, Postgresql, Hibernate. \\

\noindent \textbf{Desarrollador en Asociación Argentina de Consorcios
Regionales de Experimentación Agrícola - (AACREA), para Novamens}

\noindent Colaboré con el desarrollo del sitio web de la intranet que era
compartido a todas aquellas personas que estaba involucradas con la agricultura
a lo largo del país.  Tecnologías: Jetty, Java, Python, Cocoon, Wicket, Spring,
Jaas, Html, Xml, Xslt, Xsd, Xpath, Javascript, Ajax, Actionscript 2.0, Flash,
QT Design, Maven, Cvs, Sql, Postgresql. \\

\noindent \textbf{Desarrollador en Aerolíneas Argentinas para Novamens}

\noindent Participé en el desarrollo de una aplicación web que funcionaba como
servicio de un repositorio para la administración de datos e información
confidencial de empleados de Aerolíneas Argentinas.  Tecnologías: Tomcat, Java,
wicket, Spring, Jaas, Applet, Html, Xml, Xsd, Xpath, Javascript, Ajax, Maven,
Cvs, Sql, Postgresql. \\

\subsubsection{Habilidades}

\begin{enumerate}

  \item Experiencia en trabajo en equipos de desarrollo bajo diversas
    metodologías, Scrum como Waterfall.

  \item Experiencia en trabajo bajo condiciones de continuous integration y
    continuous delivery.

  \item Conocimiento avanzado en diversos lenguajes de programación. C++,
    Java, Groovy, Python, Clojure, LISP, Javascript, Bash.

  \item Extensiva experiencia laboral en sistemas distribuídos web, incluyendo
    arquitecturas REST, SOAP y XML-RPC, microservicios sobre HTTP y
    aplicaciones web.

  \item Experiencia laboral en desarrollo de mejoras para las aplicaciones
    mobile tanto iOS como android

  \item Experiencia laboral sobre plataformas y librerías de desarrollo para
    cada entorno, como ser Spring, JUnit, JMock, Mockito, JBoss, Tomcat para
    plataformas Java, Grails para plataforma Groovy, etc..

  \item Experiencia laboral sobre data modeling y data stores, tanto SQL
    (MySQL, Oracle, MSSQL, Postgresql) como noSQL (MongoDB, ElasticSearch,
    Redis).

\end{enumerate}

\subsubsection{Idiomas}

\begin{enumerate}

  \item Español nativo.

  \item Inglés avanzado.

  \item Italiano intermedio.

\end{enumerate}

\subsubsection{Materias aprobadas}

\begin{longtable}{|l|l|r|l|}
  \hline
  Materia                                       & Fecha      & Nota & Acta              \\
  \hline
  \endhead

  \hline
  \multicolumn{4}{r}{\textit{Continua en la próxima página}} \\
  \endfoot

  \hline
  \endlastfoot


  (6201) Física I A                                 & 14/12/2004 & 4  & 2-103-149         \\
  (7540) Algoritmos y Programación I                & 17/02/2005 & 9  & 17-92-15          \\
  (6301) Química                                    & 25/07/2005 & 6  & 3-71-228          \\
  (6103) Análisis Matemático II A                   & 16/08/2005 & 7  & 1-151-57          \\
  (6108) Álgebra II A                               & 10/02/2006 & 4  & 1-150-131         \\
  (7541) Algoritmos y Programación II               & 14/07/2006 & 7  & 17-94-186         \\
  (6203) Física II A                                & 28/12/2006 & 5  & 2-105-81          \\
  (7507) Algoritmos y Programación III              & 19/02/2007 & 5  & 17-95-235         \\
  (6109) Probabilidad y Estadística B               & 28/02/2008 & 4  & 1-152-184         \\
  (6602) Laboratorio                                & 25/07/2008 & 6  & 6-135-243         \\
  (7118) Estructura Económica Argentina             & 13/02/2009 & 7  & 11-147-226        \\
  (7512) Análisis Numérico I                        & 25/02/2009 & 10 & 17-100-212        \\
  (6670) Estructura del Computador                  & 25/02/2009 & 7  & 6-137-18          \\
  (6110) Análisis Matemático III A                  & 27/08/2009 & 7  & 1-149-214         \\
  (7542) Taller de Programación I                   & 03/02/2010 & 6  & 17-102-226        \\
  (6215) Física III D                               & 03/03/2010 & 8  & 2-108-68          \\
  (7506) Organización de Datos                      & 03/03/2011 & 7  & 17-106-86         \\
  (7509) Análisis de la Información                 & 27/06/2011 & 6  & 17-106-138        \\
  (7508) Sistemas Operativos                        & 21/07/2011 & 8  & 17-107-46         \\
  (7801) Idioma Ingles                              & 30/11/2011 & 10 & 18-23-114         \\
  (7112) Estructura de las Organizaciones           & 21/12/2011 & 7  & 11-153-133        \\
  (7510) Técnicas de Diseño                         & 21/12/2011 & 7  & 17-108-75         \\
  (6107) Matemática Discreta                        & 27/02/2013 & 6  & 1-156-136         \\
  (7523) Inteligencia Artificial                    & 18/07/2013 & 8  & 95-0001130        \\
  (7114) Modelos y Optimización I                   & 07/08/2013 & 5  & 71-0001192        \\
  (7545) Taller de Desarrollo de Proyectos I        & 19/12/2013 & 8  & 95-0001578        \\
  (7140) Leg. y ej. prof. de la ing. en informat.   & 04/07/2014 & 8  & 71-0001568        \\
  (6620) Organización de Computadoras               & 23/02/2015 & 5  & 86-0002165        \\
  (7567) Sist.Autom.de Diag.y Detec.de Fallas I     & 13/07/2015 & 8  & 95-0003040        \\
  (7515) Base de Datos                              & 05/08/2015 & 5  & 95-0003237        \\
  (7544) Adm. y Control de Proy. Informáticos I     & 21/12/2015 & 6  & 95-0003461        \\
  (7115) Modelos y Optimización II                  & 21/12/2015 & 6  & 71-0002287        \\
  (7569) Sist.Autom.de Diag.y Detec.de Fallas II    & 24/02/2016 & 7  & 95-0003599        \\
  (7531) Teoría de Lenguaje                         & 26/02/2016 & 5  & 95-0003731        \\
  (7550) Introducción a los Sistemas Inteligentes   & 29/02/2016 & 7  & 95-0003649        \\
  (7547) Taller de Desarrollo de Proyectos II       & 05/07/2016 & 8  & 95-0003836        \\

\end{longtable}

\textbf{Promedio académico}: 6,64

\subsubsection{Plan de cursada}

\textbf{Primer cuatrimestre 2016}

\begin{enumerate}
  \item (7546) Adm. y Control de Proy. Informáticos II
  \item (7547) Taller de Desarrollo de Proyectos II
  \item (7113) Información en las Organizaciones
  \item (7548) Calidad en el Desarrollo de Sistemas
\end{enumerate}

\textbf{Segundo cuatrimestre 2016}

\begin{enumerate}
  \item (7552) Taller de Programación II
\end{enumerate}

\textbf{Primer cuatrimestre 2017}

\begin{enumerate}
  \item (7543) Introducción a los Sistemas Distribuidos
  \item (7565) Manufactura Integrada por Comp. I
\end{enumerate}

%-----------------------------------------------------------------------
% Referencias
%-----------------------------------------------------------------------
\begin{thebibliography}{99}

  \bibitem{oz}
    \emph{Oz (Programming Language)},
    \url{https://en.wikipedia.org/wiki/Oz_(programming_language)},
    Wikipedia, The Free Encyclopedia

  \bibitem{ctmcp}
    \emph{Concepts, Techniques and Models of Computer Programming},
    Peter Van-Roy \& Seif Haridi,
    MIT Press,
    Enero 2004

  \bibitem{mozboard}
    \emph{Programming Languages and Distributed Computing Research Group},
    \url{https://www.info.ucl.ac.be/~pvr/pldc.html},
    Université Catholique de Louvain

  \bibitem{mozart}
    \emph{The Mozart Programming System},
    \url{http://mozart.github.io/},
    The Mozart Programming System

  \bibitem{emacs}
    \emph{GNU Emacs},
    \url{https://www.gnu.org/software/emacs/},
    The GNU Project

  \bibitem{googleoauth}
    \emph{Using OAuth 2.0 to access Google APIs},
    \url{https://developers.google.com/identity/protocols/OAuth2},
    Google Identity Platform

  \bibitem{facebookoauth}
    \emph{Inicio de sesión con Facebook},
    \url{https://developers.facebook.com/docs/facebook-login/web},
    Facebook for Developers

  \bibitem{githuboauth}
    \emph{GitHub OAuth},
    \url{https://developer.github.com/v3/oauth/},
    GitHub Developers

  \bibitem{pwa}
    \emph{Progressive Web Apps},
    \url{https://developers.google.com/web/progressive-web-apps/},
    Google Web Development

  \bibitem{chrome}
    \emph{Google Chrome},
    \url{https://www.google.com/chrome/browser/desktop/index.html},
    Google Chrome

  \bibitem{kanban}
    \emph{A brief introduction to Kanban},
    \url{https://www.atlassian.com/agile/kanban},
    Atlassian

  \bibitem{scrum}
    \emph{A brief introduction to Scrum},
    \url{https://www.atlassian.com/agile/scrum},
    Atlassian

  \bibitem{userstory}
    \emph{User stories: an agile introduction},
    \url{http://www.agilemodeling.com/artifacts/userStory.htm},
    Agile Modeling

  \bibitem{kanbanboard}
    \emph{What is a Kanban board?},
    \url{https://leankit.com/learn/kanban/kanban-board/},
    Leankit

  \bibitem{standup}
    \emph{Daily Meeting},
    \url{https://www.agilealliance.org/glossary/daily-meeting/},
    Agile Alliance

  \bibitem{retro}
    \emph{Heartbeat meeting},
    \url{https://www.agilealliance.org/glossary/heartbeatretro/},
    Scrum Alliance

\end{thebibliography}

\end{document}
